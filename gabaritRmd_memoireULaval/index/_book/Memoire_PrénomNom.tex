% This is the Reed College LaTeX thesis template. Most of the work
% for the document class was done by Sam Noble (SN), as well as this
% template. Later comments etc. by Ben Salzberg (BTS). Additional
% restructuring and APA support by Jess Youngberg (JY).
% Your comments and suggestions are more than welcome; please email
% them to cus@reed.edu
%
% See http://web.reed.edu/cis/help/latex.html for help. There are a
% great bunch of help pages there, with notes on
% getting started, bibtex, etc. Go there and read it if you're not
% already familiar with LaTeX.
%
% Any line that starts with a percent symbol is a comment.
% They won't show up in the document, and are useful for notes
% to yourself and explaining commands.
% Commenting also removes a line from the document;
% very handy for troubleshooting problems. -BTS

% As far as I know, this follows the requirements laid out in
% the 2002-2003 Senior Handbook. Ask a librarian to check the
% document before binding. -SN

%%
%% Preamble
%%
% \documentclass{<something>} must begin each LaTeX document
\documentclass[12pt]{ulaval}
% Packages are extensions to the basic LaTeX functions. Whatever you
% want to typeset, there is probably a package out there for it.
% Chemistry (chemtex), screenplays, you name it.
% Check out CTAN to see: http://www.ctan.org/
%%
\usepackage{graphicx,latexsym}
% \usepackage{amsmath}
\usepackage{amssymb,amsthm}
\usepackage{longtable,booktabs,setspace}
\usepackage{chemarr} %% Useful for one reaction arrow, useless if you're not a chem major
\usepackage[hyphens]{url}
% Added by CII
\usepackage{hyperref}
\usepackage{lmodern}
\usepackage{float}
\usepackage{tikz}
\usetikzlibrary{shapes}
\usetikzlibrary{arrows} 
\floatplacement{figure}{t}
% End of CII addition
\usepackage{rotating}

% Next line commented out by CII
%%% \usepackage{natbib}
% Comment out the natbib line above and uncomment the following two lines to use the new
% biblatex-chicago style, for Chicago A. Also make some changes at the end where the
% bibliography is included.
%\usepackage{biblatex-chicago}
%\bibliography{thesis}


% Added by CII (Thanks, Hadley!)
% Use ref for internal links
\renewcommand{\hyperref}[2][???]{\autoref{#1}}
\def\chapterautorefname{Chapter}
\def\sectionautorefname{Section}
\def\subsectionautorefname{Subsection}
% End of CII addition

% Added by CII
\usepackage{caption}
\captionsetup{width=5in}
% End of CII addition

% \usepackage{times} % other fonts are available like times, bookman, charter, palatino

% Syntax highlighting #22

% To pass between YAML and LaTeX the dollar signs are added by CII
\titre{Titre\\
\hspace*{0.333em}Sous-titre}
\auteur{Prénom Nom}
% The month and year that you submit your FINAL draft TO THE LIBRARY (May or December)
\date{Mois Année}
\directeur{Prénom Nom}
\direction{directeur}
\codirection{}
\grade{M.A.}
\diplome{Maîtrise}
\type{Mémoire}
%If you have two advisors for some reason, you can use the following
% Uncommented out by CII
% End of CII addition

%%% Remember to use the correct department!
\departement{science politique}

% Added by CII
%%% Copied from knitr
%% maxwidth is the original width if it's less than linewidth
%% otherwise use linewidth (to make sure the graphics do not exceed the margin)
\makeatletter
\def\maxwidth{ %
  \ifdim\Gin@nat@width>\linewidth
    \linewidth
  \else
    \Gin@nat@width
  \fi
}
\makeatother

\renewcommand{\contentsname}{Table des matières}
% End of CII addition

\setlength{\parskip}{0pt}

% Added by CII

\providecommand{\tightlist}{%
  \setlength{\itemsep}{0pt}\setlength{\parskip}{0pt}}

\Remerciements{
Rédigez ici même vos remerciements.

\hypertarget{section}{%
\chapter*{}\label{section}}

\hypertarget{avant-propos}{%
\chapter{Avant-propos}\label{avant-propos}}

Rédigez ici votre avant-propos.
}

\Dedication{

}

\Preface{

}

\Resume{
Rédigez ici votre résumé (en français)

\hypertarget{section-1}{%
\chapter*{}\label{section-1}}
}

\Abstract{
Rédigez ici votre abstract (en anglais).
}

	\usepackage{setspace}
 \usepackage[labelfont=bf,textfont=md]{caption}
 \usepackage{float}
 \usepackage[nottoc]{tocbibind}
 \AtBeginDocument{\renewcommand{\contentsname}{Table des matières}}
 \AtBeginDocument{\renewcommand{\listfigurename}{Liste des figures}}
 \AtBeginDocument{\renewcommand{\listtablename}{Liste des tableaux}}
 \AtBeginDocument{\renewcommand{\chaptername}{Article}}
 \usepackage{pdflscape}
 \newcommand{\blandscape}{\begin{landscape}}
 \newcommand{\elandscape}{\end{landscape}}
% End of CII addition
%%
%% End Preamble
%%
%
\begin{document}

% Everything below added by CII
  \maketitle

\frontmatter % this stuff will be roman-numbered
\pagestyle{plain} % this removes page numbers from the frontmatter
\setcounter{page}{2}
  \begin{resume}
    Rédigez ici votre résumé (en français)

    \hypertarget{section-1}{%
    \chapter*{}\label{section-1}}
  \end{resume}
  \begin{abstract}
    Rédigez ici votre abstract (en anglais).
  \end{abstract}
% 
  \hypersetup{linkcolor=black}
   \setcounter{tocdepth}{2}
  \tableofcontents

  \listoftables

  \listoffigures
  \begin{remerciements}
    Rédigez ici même vos remerciements.

    \hypertarget{section}{%
    \chapter*{}\label{section}}

    \hypertarget{avant-propos}{%
    \chapter{Avant-propos}\label{avant-propos}}

    Rédigez ici votre avant-propos.
  \end{remerciements}
% 
\mainmatter % here the regular arabic numbering starts
\pagestyle{fancyplain} % turns page numbering back on

\doublespace

\hypertarget{introduction}{%
\chapter*{Introduction}\label{introduction}}
\addcontentsline{toc}{chapter}{Introduction}

Rédiger ici l'introduction de votre mémoire.

\hypertarget{votre-titre}{%
\chapter{Votre titre}\label{votre-titre}}

\hypertarget{abstract}{%
\section{Abstract}\label{abstract}}
\begin{quote}
Rédigez ici votre abstract (en anglais).
\end{quote}
\hypertarget{ruxe9sumuxe9}{%
\section{Résumé}\label{ruxe9sumuxe9}}
\begin{quote}
Rédigez ici votre résumé (en français)
\end{quote}
\begin{quote}
\textbf{Keywords}
Écrire ici vos mots-clés
\end{quote}
\hypertarget{texte}{%
\section{Texte}\label{texte}}

Rédigez ici votre étude.

\hypertarget{references}{%
\section{References}\label{references}}

\hypertarget{appendix}{%
\section{Appendix}\label{appendix}}

\hypertarget{votre-titre-1}{%
\chapter{Votre titre}\label{votre-titre-1}}

\hypertarget{abstract-1}{%
\section{Abstract}\label{abstract-1}}
\begin{quote}
Rédigez ici votre abstract (en anglais).
\end{quote}
\hypertarget{ruxe9sumuxe9-1}{%
\section{Résumé}\label{ruxe9sumuxe9-1}}
\begin{quote}
Rédigez ici votre résumé (en français)
\end{quote}
\begin{quote}
\textbf{Keywords}
Écrire ici vos mots-clés
\end{quote}
\hypertarget{texte-1}{%
\section{Texte}\label{texte-1}}

Rédigez ici votre étude.

\hypertarget{references-1}{%
\section{References}\label{references-1}}

\hypertarget{appendix-1}{%
\section{Appendix}\label{appendix-1}}

\hypertarget{conclusion}{%
\chapter*{Conclusion}\label{conclusion}}
\addcontentsline{toc}{chapter}{Conclusion}

Rédigez ici votre conclusion.

\backmatter

\hypertarget{bibliographie}{%
\chapter*{Bibliographie}\label{bibliographie}}
\addcontentsline{toc}{chapter}{Bibliographie}

\markboth{References}{References}

\noindent

\setlength{\parindent}{-0.20in}
\setlength{\leftskip}{0.20in}
\setlength{\parskip}{8pt}


% Index?

\end{document}
